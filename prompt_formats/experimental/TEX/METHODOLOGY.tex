\documentclass{article}
\usepackage[utf8]{inputenc}
\usepackage{hyperref}
\usepackage{enumitem}

\title{Project Documentation Methodology}
\author{Jay Baleine}
\date{2025}

\begin{document}

\maketitle

\section*{License}
Disciplined AI Software Development Methodology © 2025 by Jay Baleine is licensed under CC BY-SA 4.0 \\
\url{https://creativecommons.org/licenses/by-sa/4.0/}

\subsection*{Attribution Requirements}
\begin{itemize}[noitemsep]
\item When sharing content publicly (repositories, documentation, articles): Include the full attribution above
\item When working with AI systems (ChatGPT, Claude, etc.): Attribution not required during collaboration sessions
\item When distributing or modifying the methodology: Full CC BY-SA 4.0 compliance required
\end{itemize}

\section{Core Architectural Principles}

\subsection{Foundational Philosophy}
Architectural Minimalism with Deterministic Reliability: Every line of code must earn its place through measurable value. Build systems that work predictably in production, not demonstrations of sophistication.

\subsection{Code Architecture Principles}

\subsubsection{Separation Of Concerns}
\begin{itemize}[noitemsep]
\item Each module has single, well-defined responsibility
\item Strict modular boundaries with clear interfaces
\item Recognize when separation would harm rather than help architecture
\item Centralized main entry points with modular project layout
\end{itemize}

\subsubsection{Deterministic Operations}
\begin{itemize}[noitemsep]
\item Synchronous, predictable behavior over async complexity
\item Long-runtime stability over cutting-edge patterns
\item Production stability over development convenience
\item Cross-platform considerations in design decisions
\end{itemize}

\subsubsection{Performance Driven Decisions}
\begin{itemize}[noitemsep]
\item Choose based on workload requirements, not popular trends
\item Apply optimizations only to proven bottlenecks with measurable impact
\item Avoid premature optimization that clutters codebase
\item Maintain performance baselines and regression detection
\end{itemize}

\subsubsection{Code Quality Standards}
\begin{itemize}[noitemsep]
\item Files never exceed 150 lines (split into separate modules if needed)
\item Self-explanatory code without comments
\item Preserve readability and maintainability as primary concerns
\item KISS and DRY principles expertly applied
\item Reuse existing functions before creating new ones
\end{itemize}

\subsubsection{Error Handling Philosophy}
\begin{itemize}[noitemsep]
\item Robust without over-engineering
\item Implement what's necessary for production reliability
\item Avoid handling every possible edge case
\item Graceful failure modes and resource cleanup
\end{itemize}

\subsubsection{Feature Control}
\begin{itemize}[noitemsep]
\item Resist feature bloat and complexity creep
\item Every addition must serve core project purpose
\item Surgical approach: target exact problem with minimal code
\item Multi-language use only when justified by measurable gains
\end{itemize}

\subsubsection{Web Development Adaptations}
\begin{itemize}[noitemsep]
\item No inlining: Styles to separate files, handlers to named functions, configurations as constants
\item File size accommodation: Components ≤250 lines (DOM complexity), modules ≤150 lines
\item Async operations: API calls, user interactions, data fetching only
\item Error boundaries: Network operations, user inputs, third-party integrations
\item File colocation: Component.jsx, Component.module.css, Component.test.js
\item Component splitting: Multiple purposes or testing difficulty
\item Implementation protocol: Request architectural compliance clarification for code generation tasks
\end{itemize}

\section{Phase 0 Requirements}
\textit{Basic Must-Haves (Phase 0 - Always First)}

Every project, regardless of size, must establish these foundational systems before any feature development.

\subsection{Benchmarking Suite}
\begin{itemize}[noitemsep]
\item Core Framework: Performance measurement with component isolation
\item Regression Detection: Compare against previous results, fail on performance drops
\item Baseline Management: Save and track performance baselines over time
\item JSON Output: Structured data for automated analysis and CI integration
\item Timeline Tracking: Historical performance data across project evolution
\end{itemize}

\subsection{CI/CD Infrastructure}
\begin{itemize}[noitemsep]
\item Release Workflows: Automated versioning, building, and deployment
\item Regression Detection: Benchmark comparison on every commit/PR
\item Quality Gates: Block merges that fail performance or quality thresholds
\item Automated Testing: Run full test suite on code changes
\end{itemize}

\subsection{Core Architecture}
\begin{itemize}[noitemsep]
\item Centralized Entry Points: Single main module that orchestrates everything
\item Configuration Management: Externalized settings with validation
\item Centralized Logging: Error handling and diagnostic output with JSON integration
\item Dependency Injection: Clean separation and testable components
\end{itemize}

\subsection{Testing Infrastructure}
\begin{itemize}[noitemsep]
\item Test Suite: Unit and integration tests for all components
\item Stress Testing: Load and boundary condition validation
\item Test Data Management: Reproducible test scenarios and cleanup
\item Coverage Tracking: Ensure adequate test coverage before releases
\end{itemize}

\subsection{Documentation System}
\begin{itemize}[noitemsep]
\item Automated Generation: Extract documentation from code and structure
\item Architecture Documentation: System design and component relationships
\item API Documentation: Interface specifications and usage examples
\item Performance Documentation: Benchmark results and optimization guides
\end{itemize}

\textbf{Critical Note:} These systems must be operational before writing any application logic. They become the foundation that enables rapid, confident development.

\section{Documentation Building Process}

\subsection{Step 1: Project Decomposition}
\textbf{Questions:}
\begin{itemize}[noitemsep]
\item What does "finished" look like?
\item What are the major pieces that need to exist?
\item What depends on what?
\item Where are the natural stopping points?
\end{itemize}

\textbf{Approach:} Create sections based on dependencies: Major Piece A → Major Piece B → Major Piece C with corresponding sub-tasks.

\subsection{Step 2: Phase Creation}
\textbf{Mandatory Phase 0:}
\begin{itemize}[noitemsep]
\item Benchmarking suite with regression detection
\item GitHub workflows for releases and quality gates
\item Test infrastructure (unit + stress testing)
\item Documentation generation system
\item Centralized architecture setup
\end{itemize}

\textbf{Grouping Criteria:}
\begin{itemize}[noitemsep]
\item Dependency chains: Things that must happen in sequence
\item Logical groupings: Related functionality that makes sense together
\item Natural checkpoints: Places where you can validate progress
\end{itemize}

\subsection{Step 3: Task Breakdown}
\textbf{Requirements:}
\begin{itemize}[noitemsep]
\item Specific action: What exactly needs to be done
\item Output: What will exist when complete
\item Success criteria: How to verify completion
\item Integration points: How it connects to other work
\end{itemize}

\subsection{Step 4: Progress Tracking System}
\textbf{Status Indicators:}
\begin{itemize}[noitemsep]
\item \textbf{COMPLETED:} Done and validated
\item \textbf{BLOCKED:} Cannot proceed due to dependency
\item \textbf{READY:} Dependencies met, can start
\item \textbf{UNCERTAIN:} Need clarification or decision
\end{itemize}

\subsection{Step 5: Quality Gates}
\textbf{Criteria:}
\begin{itemize}[noitemsep]
\item Does the output match what was specified?
\item Can the next phase actually use this output?
\item Is there enough documentation for future reference?
\item Are there any obvious issues that need fixing?
\end{itemize}

\section{Systematic Enforcement Framework}

\subsection{Mandatory Checkpoints}

\subsubsection{Architectural Compliance}
\begin{itemize}[noitemsep]
\item SoC VALIDATION: Each module single responsibility, clear boundaries
\item DETERMINISTIC BEHAVIOR: Synchronous operations, predictable outcomes
\item FILE SIZE COMPLIANCE: All files ≤150 lines or properly modularized
\item DRY ENFORCEMENT: No duplicate code, existing functions reused
\item KISS VALIDATION: Minimal complexity, surgical implementations
\item CONFIG CENTRALIZATION: No hardcoded values outside constants
\item PERFORMANCE INTEGRATION: Benchmarks operational, gates passing
\item PRODUCTION READINESS: Error handling, resource cleanup, cross-platform
\end{itemize}

\subsubsection{Code Quality Gates}
\begin{itemize}[noitemsep]
\item Self-explanatory naming, no comments needed
\item Performance characteristics match workload requirements
\item Every addition serves core project purpose
\item Regression detection prevents performance degradation
\item Resource utilization within defined thresholds
\end{itemize}

\textbf{Progression Blocker:} Any failed checkpoint blocks phase advancement.

\subsection{Mid-Phase Validation}

\subsubsection{During Development}
\begin{itemize}[noitemsep]
\item INCREMENTAL COMPLIANCE: Check after each significant change
\item BENCHMARK INTEGRATION: New components measured immediately
\item DEPENDENCY ALIGNMENT: Imports match architectural boundaries
\item EDGE CASE HANDLING: Document but don't implement without plan
\item FEATURE CREEP CHECK: Question necessity of each addition
\end{itemize}

\subsubsection{Before Phase Completion}
\begin{itemize}[noitemsep]
\item FULL ARCHITECTURE AUDIT: All principles systematically verified
\item PERFORMANCE REGRESSION: Compare against established baselines
\item INTEGRATION VALIDATION: Components work within system boundaries
\item PRODUCTION SIMULATION: Test under realistic deployment constraints
\end{itemize}

\subsection{Enforcement Automation}

\subsubsection{Validate Phase Script}
\begin{itemize}[noitemsep]
\item Check file sizes (fail if >150 lines)
\item Scan for hardcoded values outside config
\item Validate import dependencies match architecture
\item Run benchmark suite and check gates
\item Generate compliance report
\end{itemize}

\subsubsection{DRY Audit Script}
\begin{itemize}[noitemsep]
\item Detect duplicate function implementations
\item Find unused imports and functions
\item Identify constants that should be centralized
\item Flag potential separation of concerns violations
\end{itemize}

\subsubsection{CI/CD Workflow Integration}
\begin{itemize}[noitemsep]
\item Run validation on every commit
\item Block merges that fail compliance checks
\item Generate performance regression reports
\item Maintain baseline measurements over time
\end{itemize}

\section{Principle Integration}

\subsection{Implementation Enforcement}

\subsubsection{File And Module Constraints}
\begin{itemize}[noitemsep]
\item Each file ≤ 150 lines or properly split
\item Module serves single, clear purpose
\item No redundant code between modules
\item Existing functions reused before creating new ones
\item Naming conventions consistent across codebase
\end{itemize}

\subsubsection{Architecture Validation}
\begin{itemize}[noitemsep]
\item Centralized configuration used throughout
\item Constants referenced, no magic numbers
\item Modular separation maintained with clear boundaries
\item Dependencies align with separation of concerns
\item Synchronous operations preferred over async complexity
\end{itemize}

\subsubsection{Performance Integration}
\begin{itemize}[noitemsep]
\item Benchmarking suite integrated with all modules
\item Regression detection operational
\item JSON output for automated analysis
\item Performance gates defined and enforced
\item Timeline tracking for historical comparison
\end{itemize}

\subsubsection{Production Readiness}
\begin{itemize}[noitemsep]
\item Cross-platform deployment considerations
\item Real-world constraints addressed
\item Resource cleanup on shutdown
\item Deterministic behavior under load
\item Error handling appropriate for production
\end{itemize}

\subsection{Scaling Adaptation Guidelines}

\subsubsection{Single File Scripts}
\begin{itemize}[noitemsep]
\item Apply SoC within functions (input, processing, output)
\item Benchmark core operation even if simple
\item Validate against 150-line limit
\item Self-explanatory function and variable names
\end{itemize}

\subsubsection{Small Applications}
\begin{itemize}[noitemsep]
\item Strict modular boundaries with clear interfaces
\item Centralized configuration and constants
\item Synchronous operations with predictable flow
\item Performance baseline establishment
\end{itemize}

\subsubsection{Production Systems}
\begin{itemize}[noitemsep]
\item Full architectural compliance with all principles
\item Comprehensive benchmarking and regression detection
\item Cross-platform deployment considerations
\item Production-grade error handling and resource management
\end{itemize}

\subsubsection{Multi-Language Projects}
\begin{itemize}[noitemsep]
\item Each language justified by measurable performance gains
\item Maintain architectural principles across language boundaries
\item Unified benchmarking system for all components
\item Consistent error handling patterns across languages
\end{itemize}

\subsection{Domain Specific Adaptations}

\subsubsection{Web Development Projects}
\begin{itemize}[noitemsep]
\item No Inlining: Styles to separate files, handlers to named functions, configs as constants
\item File Size Exemption: Components ≤250 lines (DOM complexity), modules ≤150 lines
\item Async Permitted: API calls, user interactions, data fetching only
\item Error Boundaries: Network ops, user inputs, third-party integrations
\item File Colocation: Component.jsx, Component.module.css, Component.test.js
\item Component Splitting: Multiple purposes or testing difficulty
\end{itemize}

\section{Success Metrics}

\subsection{Technical Indicators}
\begin{itemize}[noitemsep]
\item All architectural principles consistently applied across codebase
\item Performance baselines maintained throughout development lifecycle
\item Zero production incidents related to architectural violations
\item File size constraints adhered to without compromising functionality
\end{itemize}

\subsection{Operational Indicators}
\begin{itemize}[noitemsep]
\item System uptime and reliability under production load
\item Predictable resource utilization patterns
\item Graceful degradation under stress conditions
\item Maintainability preserved as codebase grows
\end{itemize}

\subsection{Development Indicators}
\begin{itemize}[noitemsep]
\item Enforcement checkpoints prevent architectural drift
\item Performance regression detection catches optimizations and degradations
\item Code review efficiency improved through systematic validation
\item Technical debt accumulation prevented through continuous compliance
\end{itemize}

\subsection{Documentation Quality}
\begin{itemize}[noitemsep]
\item Enforcement checkpoints prevent architectural drift
\item Quality gates block progression with incomplete work
\item Automated validation catches compliance violations
\item Performance baselines maintained throughout development
\end{itemize}

\subsection{Project Execution}
\begin{itemize}[noitemsep]
\item Systematic validation prevents technical debt accumulation
\item Architectural principles consistently applied across codebase
\item Performance characteristics predictable and measurable
\item Production readiness verified at each phase
\end{itemize}

\section{Conclusion}
This methodology enforces discipline through automated checking and explicit validation points, preventing the gradual erosion of architectural principles during development.

\end{document}